\documentclass[12pt,american,english, pointlessnumbers, abstracton, headsepline]{scrreprt}

\renewcommand{\familydefault}{\sfdefault}

%----------------------------------------------------------------------------------------------------------------------------------------------------------------------
% encoding
%----------------------------------------------------------------------------------------------------------------------------------------------------------------------
\usepackage[T1]{fontenc}
\usepackage[utf8]{inputenc}

\usepackage{geometry}

\geometry{verbose,tmargin=2.5cm,bmargin=2.5cm}
\pagestyle{plain}
\setlength{\parskip}{\medskipamount}
% \setlength{\parindent}{0pt}

\usepackage{color}
\usepackage{xcolor}
\usepackage[english]{babel}
\usepackage{array}
\usepackage{varioref}
\usepackage{textcomp}
\usepackage{multirow}
\usepackage{amsmath}
\usepackage{amsthm}
\usepackage{graphicx}
\usepackage{setspace}
\usepackage{nomencl}
\usepackage{tkz-graph}
\usepackage[hyphens]{url}
\usepackage{listings}
\usepackage{mdframed}
\usepackage{float}
\usepackage{alltt}
\usepackage[nounderscore]{syntax}
\usepackage{slashbox}

%----------------------------------------------------------------------------------------------------------------------------------------------------------------------
% nicer tables
%----------------------------------------------------------------------------------------------------------------------------------------------------------------------
\usepackage{booktabs}
\newcommand{\ra}[1]{\renewcommand{\arraystretch}{#1}}

%\usepackage[acronym]{glossaries}

%\newglossary{abbrev}{abs}{abo}{List of Abbreviations}
%
%\newglossaryentry{MS}{
%    name        = MS ,
%    description = mass spectroscopy ,
%    type        = abbrev
%}

%\makeglossaries

%----------------------------------------------------------------------------------------------------------------------------------------------------------------------
% syntax trees
%----------------------------------------------------------------------------------------------------------------------------------------------------------------------
\usepackage{qtree}

\setlength{\nomlabelwidth}{.20\hsize}
\renewcommand{\nomlabel}[1]{#1 \dotfill}

%\setlength{\nomitemsep}{-\parsep}

\makenomenclature

\setstretch{1.2}

\makeatletter

% Because html converters don't know tabularnewline
\providecommand{\tabularnewline}{\\}

% verschieden Symbole, Zeichen wie (c), �
\usepackage{textcomp,units}

%  more space between table and subtitle
\usepackage[tableposition=top]{caption}
\captionsetup[table]{skip=10pt}

%----------------------------------------------------------------------------------------------------------------------------------------------------------------------
% big chapter number indicator
%----------------------------------------------------------------------------------------------------------------------------------------------------------------------
\makeatletter
 \renewcommand*{\chapterformat}{% 
   \begingroup
     \setlength{\unitlength}{1mm}% 
     \begin{picture}(10,10)(0,5)
       \setlength{\fboxsep}{0pt}
       %\put(0,0){\framebox(20,40){}}% 
       %\put(0,20){\makebox(20,20){\rule{20\unitlength}{20\unitlength}}}% 
       \put(10,15){\line(1,0){\dimexpr 
           \textwidth-20\unitlength\relax\@gobble}}% 
       \put(0,0){\makebox(10,20)[r]{% 
           \fontsize{28\unitlength}{28\unitlength}\selectfont\thechapter 
           \kern-.05em% Ziffer in der Zeichenzelle nach rechts schieben 
         }}% 
       \put(10,15){\makebox(\dimexpr 
           \textwidth-20\unitlength\relax\@gobble,\ht\strutbox\@gobble)[l]{% 
             \ \normalsize\color{black}\chapapp~\thechapter\autodot 
           }}% 
     \end{picture} % <-- Leerzeichen ist hier beabsichtigt! 
   \endgroup 
}

\usepackage[automark]{scrpage2}
%\automark[chapter]{chapter}
\clearscrheadfoot
\ohead{\\\headmark}
\ihead{\includegraphics[scale=0.15]{logo.jpg}}%\pagemark}
\ofoot[\pagemark]{\pagemark}


% summary and abstract (english) on one page
\renewenvironment{abstract}{
    \@beginparpenalty\@lowpenalty
      \begin{center}
        \normalfont\sectfont\nobreak\abstractname
        \@endparpenalty\@M
      \end{center}
}{
    \par
}

%----------------------------------------------------------------------------------------------------------------------------------------------------------------------
% appearance optimization (commented out for faster compilation)
%----------------------------------------------------------------------------------------------------------------------------------------------------------------------
% optimization of appearance
\usepackage{microtype}
\usepackage{ %a4wide,
            ellipsis, fixltx2e, mparhack,
            booktabs, longtable
} 

\usepackage{ifpdf} % part of the hyperref bundle
\ifpdf
	%set fonts for nicer pdf view
	 \IfFileExists{lmodern.sty}{\usepackage{lmodern}}
	  {\usepackage[scaled=0.92]{helvet}
	    \usepackage{mathptmx}
	    \usepackage{courier} }
\fi

 % the pages of the TOC are numbered roman
 % and a pdf-bookmark for the TOC is added
 \pagenumbering{roman}
 \let\myTOC\tableofcontents
 \renewcommand\tableofcontents{
   %\pdfbookmark[1]{Contents}{}
   \myTOC
   \clearpage
   \pagenumbering{arabic}}

% cross refs as links
 \usepackage[colorlinks=true, bookmarks, bookmarksnumbered, bookmarksopen, bookmarksopenlevel=1,
  linkcolor=black, citecolor=black, urlcolor=blue, filecolor=blue,
  pdfpagelayout=OneColumn, pdfnewwindow=true,
  pdfstartview=XYZ, plainpages=false, pdfpagelabels,
  pdfauthor={LyX Team}, pdftex,
  pdftitle={LyX's Figure, Table, Floats, Notes, and Boxes manual},
  pdfsubject={LyX-documentation about figures, tables, floats, notes, and boxes},
  pdfkeywords={LyX, Tables, Figures, Floats, Boxes, Notes}]{hyperref}

% increased space between heading and table
\newcommand{\@ldtable}{}
\let\@ldtable\table
\renewcommand{\table}{ %
                 \setlength{\@tempdima}{\abovecaptionskip} %
                 \setlength{\abovecaptionskip}{\belowcaptionskip} %
                 \setlength{\belowcaptionskip}{\@tempdima} %
                 \@ldtable}

\renewcommand{\nomname}{Glossar}

\addto\captionsenglish{
  \renewcommand{\contentsname}
    {Table of Contents}
}

%----------------------------------------------------------------------------------------------------------------------------------------------------------------------
% colors
%----------------------------------------------------------------------------------------------------------------------------------------------------------------------
\definecolor{lightgray}{rgb}{0.8,0.8,0.8}

\definecolor{mygreen}{rgb}{0, 0.6, 0}
\definecolor{mygray}{rgb}{0.5, 0.5, 0.5}

\definecolor{numberbg}{rgb}{0.75, 0.75, 0.75}
\definecolor{numbercolor}{rgb}{0, 0, 0}
\definecolor{ballblue}{rgb}{0.13, 0.67, 0.8}

\AtBeginDocument{
  \def\labelitemiii{\(\circ\)}
}

\makeatother

\usepackage{listings}
\addto\captionsamerican{\renewcommand{\lstlistingname}{Listing}}
\addto\captionsenglish{\renewcommand{\lstlistingname}{Listing}}
\renewcommand{\lstlistingname}{Listing}

\makeindex

%----------------------------------------------------------------------------------------------------------------------------------------------------------------------
% quotations
%----------------------------------------------------------------------------------------------------------------------------------------------------------------------
\usepackage{epigraph}
\setlength\epigraphwidth{8cm}
\setlength\epigraphrule{0pt}

\usepackage{etoolbox}

%\makeatletter
%\patchcmd{\epigraph}{\@epitext{#1}}{\itshape\@epitext{#1}}{}{}
%\makeatother

\newcommand{
	\myquote
}[2]{
	\begin{mdframed}
		\vspace*{\fill}
		\epigraph{
			``#1''
		}{
			---#2
		}
	\end{mdframed}
}

%----------------------------------------------------------------------------------------------------------------------------------------------------------------------
% listings
%----------------------------------------------------------------------------------------------------------------------------------------------------------------------
\lstset{
	numbers=left,
	numbersep=8pt,
	numberstyle=\small\color{numbercolor},
	rulecolor=\color{numberbg},
	commentstyle=\color{mygreen},
	frame=single,
	framesep=0.5mm,
	framexleftmargin=15pt,
	fillcolor=\color{ballblue},
	tabsize=4,
	keywordstyle=\color{blue},
	morekeywords={*, SKIP, IF, THEN, ELSE, FI, WHILE, DO, OD},
	captionpos=b,
}

%definitions
\newtheorem{definition}{Definition}